%% Automaigc titlecase
\Addlcwords{for a is but and with of in as the etc on to if}

% TODO I'd rather like to hide these internal things, but I guess I'll
% settle for making them @ protected
\makeatletter
\LetLtxMacro{\@title}{\title}
\LetLtxMacro{\@section}{\section}
\LetLtxMacro{\@subsection}{\subsection}
\LetLtxMacro{\@subsubsection}{\subsubsection}

\newcommand{\@titlecrafter}[3]{
  \ifthenelse{\equal{#2}{}}
  {#1[#3]{\titlecap{#3}}}
  {#1[#2]{\titlecap{#3}}}
}

\renewcommand{\title}[2][]{
  \@titlecrafter{\@title}{#1}{#2}
}

\renewcommand{\section}[2][]{
  \@titlecrafter{\@section}{#1}{#2}
}
\renewcommand{\subsection}[2][]{
  \@titlecrafter{\@subsection}{#1}{#2}
}
\renewcommand{\subsubsection}[2][]{
  \@titlecrafter{\@subsubsection}{#1}{#2}
}
\makeatother

\newcommand{\todoleft}[2][]{
  \reversemarginpar
  \todo[#1]{#2}
  \reversemarginpar
}

\newcommand{\margincomment}[2][color=green!40,nolist]{
  \todo[#1]{#2}
}

\newcommand{\inlinecomment}[2][inline,color=green!40,nolist]{
  \todo[#1]{#2}
}

%% Misc commands <<<

\newcommand{\secref}[1]{\S\ref{#1}}
\newcommand{\figref}[1]{Figure~\ref{#1}}
\newcommand{\lemref}[1]{Lemma~\ref{#1}}
\newcommand{\thmref}[1]{Theorem~\ref{#1}}

% Text fonts
\newcommand{\tbf}[1]{\textbf{#1}}
\newcommand{\trm}[1]{\textrm{#1}}
\newcommand{\tnl}[1]{\small{\trm{#1}}}

% Math fonts
\newcommand{\mbb}[1]{\mathbb{#1}}
\newcommand{\mbf}[1]{\mathbf{#1}}
\renewcommand{\mit}[1]{\mathit{#1}}
\newcommand{\mrm}[1]{\mathrm{#1}}
\newcommand{\mtt}[1]{\mathtt{#1}}
\newcommand{\mcal}[1]{\mathcal{#1}}
\newcommand{\mfrak}[1]{\mathfrak{#1}}
\newcommand{\msf}[1]{\mathsf{#1}}
\newcommand{\mscr}[1]{\mathscr{#1}}
\renewcommand{\b}{\boldsymbol}

% Text mode
\newenvironment{nop}{}{}

% Math mode
\newenvironment{sdisplaymath}{
\begin{nop}\small\begin{displaymath}}{
\end{displaymath}\end{nop}\ignorespacesafterend}
\newenvironment{fdisplaymath}{
\begin{nop}\footnotesize\begin{displaymath}}{
\end{displaymath}\end{nop}\ignorespacesafterend}
\newenvironment{smathpar}{
\begin{nop}\small\begin{mathpar}}{
\end{mathpar}\end{nop}\ignorespacesafterend}
\newenvironment{fmathpar}{
\begin{nop}\footnotesize\begin{mathpar}}{
\end{mathpar}\end{nop}\ignorespacesafterend}
\newenvironment{alignS}{
\begin{nop}\begin{align}}{
\end{align}\end{nop}\ignorespacesafterend}
\newenvironment{salignS}{
\begin{nop}\small\begin{align}}{
\end{align}\end{nop}\ignorespacesafterend}
\newenvironment{falignS}{
\begin{nop}\footnotesize\begin{align*}}{
\end{align}\end{nop}\ignorespacesafterend}

% Stack formatting
\newenvironment{stackAux}[2]{%
\setlength{\arraycolsep}{0pt}
\begin{array}[#1]{#2}}{
\end{array}}
\newenvironment{stackCC}{
\begin{stackAux}{c}{c}}{\end{stackAux}}
\newenvironment{stackCL}{
\begin{stackAux}{c}{l}}{\end{stackAux}}
\newenvironment{stackTL}{
\begin{stackAux}{t}{l}}{\end{stackAux}}
\newenvironment{stackTR}{
\begin{stackAux}{t}{r}}{\end{stackAux}}
\newenvironment{stackBC}{
\begin{stackAux}{b}{c}}{\end{stackAux}}
\newenvironment{stackBL}{
\begin{stackAux}{b}{l}}{\end{stackAux}}

%% Theorems
\newtheoremstyle{athm}{\topsep}{\topsep}%
     {\itshape}%         Body font
     {}%         Indent amount (empty = no indent, \parindent = para indent)
     {\bfseries}% Thm head font
     {}%        Punctuation after thm head
     {\newline}%     Space after thm head (\newline = linebreak)
     {\thmname{#1}\thmnumber{ #2}\thmnote{~\,(#3)}}%         Thm head spec

\theoremstyle{athm}
\newtheorem{theorem}{Theorem}[section]
\newtheorem{lemma}[theorem]{Lemma}
\newtheorem{conjecture}[theorem]{Conjecture}
\newtheorem{corollary}[theorem]{Corollary}
\newtheorem{definition}[theorem]{Definition}

\renewenvironment{proof}[1][]{%
\noindent%
\textbf{Proof\if\relax\detokenize{#1}\relax\else~(#1)\fi}%
\begin{list}{}{}\item[]}{%
\qed
\end{list}}

\renewenvironment{cases}{\begin{description}}{\end{description}}
\newcommand{\case}[1][]{\item[Case] #1}
\newcommand{\scase}[1][]{\item[Sub-case] #1}
\newcommand{\sscase}[1][]{\item[SubSub-case] #1}

%% TODO Or these
%\newsavebox{\DetailsBox}%
%\newenvironment{details}[1][]{%
%\ifthenelse{\boolean{ShowDetailsFlag}}{}{#1 \begin{lrbox}{\DetailsBox}\begin{minipage}{\textwidth}}%
%}{%
%\ifthenelse{\boolean{ShowDetailsFlag}}{}{\end{minipage}\end{lrbox}}%
%}%

%% \renewcommand{\thefigure}{\thesection.\arabic{figure}}

% % Shading
% \newlength{\ShadeLen}
% \newcommand{\shade}[2]{%
% \settoheight{\ShadeLen}{$\bnfalt$}
% \colorbox[gray]{#1}{\mbox{\rule{0pt}{\ShadeLen}#2}}}

%% Meta Language
\newcommand{\metafont}[1]{\mathrm{#1}}
\newcommand{\dom}[1]{\metafont{dom}(#1)}
\newcommand{\cod}[1]{\metafont{cod}(#1)}
\newcommand{\rng}[1]{\metafont{rng}(#1)}
\newcommand{\FV}{\metafont{fv}}
\newcommand{\FTV}{\metafont{ftv}}
\newcommand{\subst}[3]{{#1}[{#2}/{#3}]}
\newcommand{\defeq}{\stackrel{\metafont{def}}{=}}
\newcommand{\finmap}{\stackrel{\metafont{fin}}{\rightarrow}}
\renewcommand{\iff}{\metafont{iff}}
\newcommand{\lsem}{\left\llbracket}
\newcommand{\rsem}{\right\rrbracket}
\newcommand{\sembrace}[1]{\lsem{#1}\rsem}
\newcommand{\powset}[1]{\mathscr{P}(#1)}
\newcommand{\irred}[1]{\metafont{irred}(#1)}
\renewcommand{\max}[2]{\metafont{max}(#1,#2)}
\newcommand{\free}[2]{\metafont{free}(#1,#2)}
\newcommand{\running}[2]{\metafont{running}({#1},{#2})}
\newcommand{\pair}[2]{( #1, #2 )}
\newcommand{\triple}[3]{( #1, #2, #3 )}

\newcommand{\satisfy}{\vDash}

\newcommand{\plus}{+}
\newcommand{\etal}{\textit{et al.}}
\newcommand{\bump}{\hspace{3.5pt}}
\newcommand{\fresh}[1]{(\mit{fresh}\:#1)}
\newcommand{\where}[1]{\mrm{where}\:#1}

\newcommand{\lang}[1]{\mrm{\trm{#1}}}

%% Standard symbols

\newcommand{\empctx}{\cdot}
\newcommand{\hole}{[\cdot]}
\newcommand{\hw}[1]{\lbrack{#1}\rbrack}
\newcommand{\ectxt}{E}
\newcommand{\ctxt}{C}

\newcommand{\hooklongrightarrow}{\lhook\joinrel\longrightarrow}
\newcommand{\redexstep}{\hookrightarrow}

\newcommand{\step}{\longmapsto}
\newcommand{\stepin}[1]{\step^{#1}}
\newcommand{\stepstar}{\stepin{*}}

\newcommand{\red}{\Downarrow}
\newcommand{\diverg}[1]{#1 \Uparrow}

\newcommand{\termin}[1]{#1\red}
\newcommand{\terminw}[2]{\termin{#1} #2}

\newcommand{\transarrow}{\leadsto}
\newcommand{\backtransarrow}{\twoheadrightarrow}

\newcommand{\funarrow}{\rightarrow}
\newcommand{\ctxarrow}{\Rightarrow}

% BNF symbols
\newcommand{\bnfalt}{{\bf \,\,\mid\,\,}}
\newcommand{\bnfdef}{{\bf ::=}}

%% Language Formatting combinators

% Copy-pasta: http://tex.stackexchange.com/questions/16189/repeat-command-n-times
\makeatletter
\newcommand{\Repeat}[1]{%
  \expandafter\@Repeat\expandafter{\the\numexpr #1\relax}%
}

\def\@Repeat#1{%
  \ifnum#1>0
    \expandafter\@@Repeat\expandafter{\the\numexpr #1-1\expandafter\relax\expandafter}%
  \else
    \expandafter\@gobble
  \fi
}
\def\@@Repeat#1#2{%
  \@Repeat{#1}{#2}{#2}
}
\makeatother

% \metavar formats a language meta-var.
%
% #1 : a pre-formatted symbol representing the meta-variable
% #2 : a pre-formatted subscript
% #3 : a pre-formatted superscript
% #4 : a pre-formatted prime symbol
% #5 : a natural number, representing the number of primes
%
% Usage:
%   \newcommand{\txmetavar}[3]{
%     \metavar{\tfont{x}}{\tcolor{#1}}{\tcolor{#2}}{\tprime}{#3}
%   }
%   \newcommand{\tx}{\txmetavar{}{}{}}
%   \newcommand{\txone}{\txmetavar{1}{}{}}
%   \newcommand{\txonepr}{\txmetavar{1}{}{1}}

\newcommand{\metavar}[5]{
  {#1}_{#2}^{{#3}\Repeat{#5}{#4}}
}

% \metavarto formats a language meta-var with only a superscript.
%
% #1 : a pre-formatted symbol representing the meta-var
% #2 : a pre-formatted superscript
%
% Usage:
%   \newcommand{\txF}{\metavarto{\tx}{f}}
\newcommand{\metavarto}[2]{\metavar{#1}{}{#2}{}{0}}

% \metavarin formats a language meta-var with only a subscript.
%
% #1 : a pre-formatted symbol representing the meta-var
% #2 : a pre-formatted subscript
%
% Usage:
%   \newcommand{\txone}{\metavarin{\tx}{\tcolor{1}}}
\newcommand{\metavarin}[2]{\metavar{#1}{#2}{}{}{0}}

% \metavarpr formats a language meta-var with only primes, takes 3
% parameters:
%
% #1 : a pre-formatted symbol representing the meta-var
% #2 : a pre-formatted prime symbol
% #3 : a natural number representing the number of primes
%
% Usage:
%   \newcommand{\txpr}{\metavarpr{\tx}{\prime}{1}}
%   \newcommand{\txdubpr}{\metavarpr{\tx}{\prime}{2}}
\newcommand{\metavarpr}[3]{\metavarpr{#1}{}{}{#2}{#3}}

\makeatletter

% \@globalshadowedcommand generates a new global macro. However, when
% the macro is already defined, it warns the user via a compile time
% message that the macro has been shadowed.
%
% #1 : A default value for the optional argument of the new macro
% #2 : The name of the macro to be defined
% #3 : The number of arguments the macro takes
% #4 : The body of the macro to be defined
%
% Usage:
%  \@globalshadowedcommand{\ae}{0}{
%    \metavar{\tfont{x}}{}{}{}{0}
%  }
\newcommand{\@globalshadowedcommand}[4][]{
  {
    \let\def\gdef
    \ifcsname#2\endcsname
    \message{
      ^^J^^J
      !!!!!!!!!!!!!!!!!!!!!!!!!!!!!!!!!!!!!!!!!!!!!!!!!!!!!!!!!!!!!!^^J
      !! secctex Warning: shadowed \@backslashchar#2 when defining metavar macros.^^J
      !! If you need this macro, use \@backslashchar LetLtxMacro to save it.^^J
      !!!!!!!!!!!!!!!!!!!!!!!!!!!!!!!!!!!!!!!!!!!!!!!!!!!!!!!!!!!!!!^^J
      ^^J
    }
    \let\newcommand\renewcommand
    \fi
    \if\relax\detokenize{#1}\relax
    \expandafter\newcommand\csname #2\endcsname[#3]{#4}
    \else
    \expandafter\newcommand\csname #2\endcsname[#3][#1]{#4}
    \fi
  }
}

% \newmetavarStar generates some standard macros for formatting a meta-var,
% and takes 5 paraemters:
%
% #1 : A formatting macro for the meta-var, such a \tfont
% #2 : A formatting macro for the subscripts, superscripts, and primes,
%   such as \tcolor
% #3 : A string prefix for the name of the macro, such as t
% #4 : A string name of the macro, such as ty
% #5 : A string to format and display this meta-variable as, such as
%   \tau
%
% Usage:
%   \newmetavarStar{\tfont}{\tcolor}{t}{ty}{\tau}
%
% This usage generates the following definitions:
%
%  \newcommand{\ttymetavar}[3]{
%    \metavar{\tfont{\tau}}{\tcolor{#1}}{\tcolor{#2}}{\tcolor{\prime}}{#3}
%  }
%
%  \newcommand{\tty}{ \ttymetavar{}{}{} }
%  \newcommand{\ttyin}[1]{ \ttymetavar{#1}{}{0} }
%  \newcommand{\ttyto}[1]{ \ttymetavar{}{#1}{0} }
%  \newcommand{\ttypr}[1][1]{ \ttymetavar{}{}{#1} }
%  \newcommand{\ttyone}{ \ttymetavar{1}{}{} }
%  \newcommand{\ttyonepr}[1][1]{ \ttymetavar{1}{}{#1}}
%  \newcommand{\ttytwo}{ \ttymetavar{2}{}{} }
%  \newcommand{\ttytwopr}{ \ttymetavar{2}{}{1} }
%  \newcommand{\ttyi}{ \ttymetavar{i}{}{} }
%  \newcommand{\ttyipr}[1][1]{ \ttymetavar{i}{}{#1} }
%  \newcommand{\ttyn}{ \ttymetavar{n}{}{} }
%  \newcommand{\ttynpr}[1][1]{ \ttymetavar{n}{}{#1} }
%
%   \ttymetavar takes an unformatted sub-script, super-script, and
%   a natural number n indicating the number of primes. It then formats
%   the sub-script, the super-script, and n primes and attaches them to
%   the formatted meta-variable.
\newcommand{\newmetavarStar}[5]{
  \@globalshadowedcommand{#3#4metavar}{3}{
    \metavar{#1{#5}}{#2{##1}}{#2{##2}}{#2{\prime}}{##3}
  }
  \@globalshadowedcommand{#3#4}{0}{
    \csname #3#4metavar\endcsname{}{}{0}
  }
  \@globalshadowedcommand{#3#4in}{1}{
    \csname #3#4metavar\endcsname{##1}{}{0}
  }
  \@globalshadowedcommand[0]{#3#4to}{2}{
    \csname #3#4metavar\endcsname{}{##2}{##1}
  }
  \@globalshadowedcommand[1]{#3#4pr}{1}{
    \csname #3#4metavar\endcsname{}{}{##1}
  }
  \@globalshadowedcommand{#3#4one}{0}{
    \csname #3#4metavar\endcsname{1}{}{0}
  }
  \@globalshadowedcommand[1]{#3#4onepr}{1}{
    \csname#3#4metavar\endcsname{1}{}{##1}
  }
  \@globalshadowedcommand{#3#4two}{0}{
    \csname #3#4metavar\endcsname{2}{}{0}
  }
  \@globalshadowedcommand[1]{#3#4twopr}{1}{
    \csname #3#4metavar\endcsname{2}{}{##1}
  }
  \@globalshadowedcommand{#3#4n}{0}{
    \csname #3#4metavar\endcsname{n}{}{0}
  }
  \@globalshadowedcommand[1]{#3#4npr}{1}{
    \csname #3#4metavar\endcsname{n}{}{##1}
  }
  \@globalshadowedcommand{#3#4i}{0}{
    \csname #3#4metavar\endcsname{i}{}{0}
  }
  \@globalshadowedcommand[1]{#3#4ipr}{1}{
    \csname #3#4metavar\endcsname{i}{}{##1}
  }
}

% \newmetavarNoStar is like \netmetavarStar, but assumes the name of
% the macro is also the display symbol.
% It takes 4 paraemters:
%
% #1 : A formatting macro for the meta-var, such a \tfont
% #2 : A formatting macro for the subscripts, superscripts, and primes,
%   such as \tcolor
% #3 : A string prefix for the name of the macro, such as t
% #4 : A string name of the macro and symbol to format and display, such
%   as x
%
% Usage:
%   \newmetavarNoStar{\tfont}{\tcolor}{t}{x}
%
% This usage is equivalent to
%   \newmetavarStar{\tfont}{\tcolor}{t}{x}{x}
\newcommand{\newmetavarNoStar}[4]{
  \newmetavarStar{#2}{#3}{#4}{#1}{#1}
}

\newcommand{\@newmetavarmaybestar}{
  \@ifstar
  \newmetavarStar
  \newmetavarNoStar
}

% \newmetavar implement the * LaTeX idiom after currying. It expands in
% to either \newmetavarStar or \newmetavarNoStar depending on whether the
% character following its 3rd argument is a * or not.
%
% Usage:
%   \newcommand{\newtmetavar}{\newmetavar{\tfont}{\tcolor}{t}}
%   \newtmetavar{x}
%   \newtmetavar*{ty}{\tau}
\newcommand{\newmetavar}[4]{
  \@newmetavarmaybestar#4{#1}{#2}{#3}
}

% \newmetavarS defines new metavar macros for a list of name/symbol
% pairs. Essentially, it calls \newmetavar* for each name/symbol pair
% in its 4th parameter.
%
% #1 : A formatting macro for the meta-var, such a \tfont
% #2 : A formatting macro for the subscripts, superscripts, and primes,
%   such as \tcolor
% #3 : A string prefix for the name of the macro, such as t
% #4 : A list of name/symbol pairs, separated literally by a /.
%
% Usage:
%   \newmetavarsS{\tfont}{\tcolor}{t}{x/x,e/e,v/v,alpha/\b{\alpha}}
%   \newcommand{\newsmetavarsS}{\newmetavarsS{\sfontsym}{\scolor}{s}}
%   \newsmetavarsS{\alpha/\alpha, ty/\sigma}
\newcommand{\newmetavarsS}[4]{
  \foreach \@n/\@d in {#4} {
    \edef \@meow{\noexpand\newmetavarStar{\noexpand#1}{\noexpand#2}{#3}{\@n}{\@d}}
    \@meow
  }
  \def\@meow{\undefined}
  \def\@n{\undefined}
  \def\@d{\undefined}
}

% \newmetavar defines new metavar macros for a list of names, assuming
% each metavar will be displayed via the symbol it is named.
% Essentially, it calls \newmetavar for each name in its 4th parameter.
%
% #1 : A formatting macro for the meta-var, such a \tfont
% #2 : A formatting macro for the subscripts, superscripts, and primes,
%   such as \tcolor
% #3 : A string prefix for the name of the macro, such as t
% #4 : A list of names.
%
% Usage:
%   \newmetavars{\tfont}{\tcolor}{t}{x,e,v}
%   \newcommand{\newsmetavars}{\newmetavars{\sfont}{\scolor}{s}}
%   \newsmetavars{x,e,v}
\newcommand{\newmetavars}[4]{
  \foreach \@n in {#4} {
    \edef \@meow{\noexpand\newmetavarStar{\noexpand#1}{\noexpand#2}{#3}{\@n}{\@n}}
    \@meow
  }
  \def\@meow{\undefined}
  \def\@n{\undefined}
}

% TODO Figure out that currying issue with lists

%%% ----------------- Standard language features ----------------------

%% ----------------------- Standard types ----------------------------

% \@funty formats a function type.
%
% #1 : A formatting macro for symbols, such as \tfontsym
% #2 : The pre-formatted argument to the function
% #3 : The pre-formatted result of the function
\newcommand{\@funty}[3]{#2 #1{\funarrow} \, #3}

% \funty is like \@funty but provides a interface.
%
% #1 : A formatting macro for symbols, such as \tfontsym
% #2 : A formatting macro for text, such as \tfont
% #3 : The pre-formatted argument to the function
% #4 : The pre-formatted result of the function
\newcommand{\funty}[2]{\@funty{#1}}

% \@polyfunty formats a polymorphic function type
%
% #1 : A formatting macro for symbols, such as \tfontsym
% #2 : The pre-formatted type-variable to bind
% #3 : The pre-formatted argument to the function
% #4 : The pre-formatted result of the function
\newcommand{\@polyfunty}[4]{#1{\forall}\,#1{[}{#2}#1{].}{#3}#1{\funarrow}\,{#4}}

% \polyfunty is like \@polyfunty but provides a standard interface.
%
% #1 : A formatting macro for symbols, such as \tfontsym
% #2 : A formatting macro for text, such as \tfont
% #3 : The pre-formatted type-variable to bind
% #4 : The pre-formatted argument to the function
% #5 : The pre-formatted result of the function
\newcommand{\polyfunty}[2]{\@polyfunty{#1}}

% \@forallty formats a polymorphic type.
%
% #1 : A formatting macro for symbols, such as \tfontsym
% #2 : The pre-formatted type-variable to bind
% #3 : The pre-formatted result in which the type-variable is bound
\newcommand{\@forallty}[3]{#1{\forall}#2#1{.}\,#3}

% \forallty is like \@forallty, but provides a standard interface
%
% #1 : A formatting macro for symbols, such as \tfontsym
% #2 : A formatting macro for text, such as \tfont
% #3 : The pre-formatted type-variable to bind
% #4 : The pre-formatted result in which the type-variable is bound
\newcommand{\forallty}[2]{\@forallty{#1}}

% \@existty formats an existential type.
%
% #1 : A formatting macro for symbols, such as \tfontsym
% #2 : The pre-formatted type-variable to bind
% #3 : The pre-formatted result in which the type-variable is bound
\newcommand{\@existty}[3]{#1{\exists}#2#1{.}\,#3}

% \existty is like \@existty but provides a standard interface.
%
% #1 : A formatting macro for symbols, such as \tfontsym
% #2 : A formatting macro for text, such as \tfont
% #3 : The pre-formatted type-variable to bind
% #4 : The pre-formatted result in which the type-variable is bound
\newcommand{\existty}[2]{\@existty{#1}}

% \@muty formats a recursive type.
%
% #1 : A formatting macro for symbols, such as \tfontsym
% #2 : The pre-formatted type-variable to bind
% #3 : The pre-formatted result in which the type-variable is bound
\newcommand{\@muty}[3]{#1{\upmu} #2 #1{.} #3 }

% \muty is like \@muty but provides a standard interface.
%
% #1 : A formatting macro for symbols, such as \tfontsym
% #2 : A formatting macro for text, such as \tfont
% #3 : The pre-formatted type-variable to bind
% #4 : The pre-formatted result in which the type-variable is bound
\newcommand{\muty}[2]{\@muty{#1}}

% \@unitty formats a unit type.
%
% #1 : A formatting macro for text, such as \tfont
\newcommand{\@unitty}[1]{#1{1}}

% \unitty is like \@unitty but provides a standard interface.
%
% #1 : A formatting macro for symbols, such as \tfontsym
% #2 : A formatting macro for text, such as \tfont
\newcommand{\unitty}[2]{\@unitty{#2}}

% \@voidty formats a void type.
%
% #1 : A formatting macro for text, such as \tfont
\newcommand{\@voidty}[1]{#1{0}}

% \voidty is like \@voidty but provides a standard interface.
%
% #1 : A formatting macro for symbols, such as \tfontsym
% #2 : A formatting macro for text, such as \tfont
\newcommand{\voidty}[2]{\@voidty{#2}}

% \@boolty formats a bool type.
%
% #1 : A formatting macro for text, such as \tfont
\newcommand{\@boolty}[1]{#1{bool}}

% \boolty is like \@boolty but provides a standard interface
%
% #1 : A formatting macro for symbols, such as \tfontsym
% #2 : A formatting macro for text, such as \tfont
\newcommand{\boolty}[2]{\@boolty{#2}}

% \@pairty formats a pair type.
%
% #1 : A formatting macro for symbols, such as \tfontsym
% #2 : The pre-formatted type for the first component of the pair
% #3 : The pre-formatted type for the second component of the pair
\newcommand{\@pairty}[3]{#2 \, #1{\times} \, #3}

% \pairty is like \@pairty but provides a standard interface.
%
% #1 : A formatting macro for symbols, such as \tfontsym
% #2 : A formatting macro for text, such as \tfont
% #3 : The pre-formatted type for the first component of the pair
% #4 : The pre-formatted type for the second component of the pair
\newcommand{\pairty}[2]{\@pairty{#1}}

% \@sumty formats a sum type.
%
% #1 : A formatting macro for symbols, such as \tfontsym
% #2 : The pre-formatted type for the first component of the sum
% #3 : The pre-formatted type for the second component of the sum
\newcommand{\@sumty}[3]{#2 \, #1{\plus} \, #3}

% \sumty is like \@sumty but provides a standard interface.
%
% #1 : A formatting macro for symbols, such as \tfontsym
% #2 : A formatting macro for text, such as \tfont
% #3 : The pre-formatted type for the first component of the sum
% #4 : The pre-formatted type for the second component of the sum
\newcommand{\sumty}[2]{\@sumty{#1}}

%% Standard expression

% \@fune formats a function expression.
%
% #1 : A formatting macro for symbols, such as \tfontsym.
% #2 : The pre-formatted variable the function binds.
% #3 : The pre-formatted type of the variable.
% #4 : The pre-formatted body of the function.
\newcommand{\@fune}[4]{#1{\uplambda (}{#2}\,#1{:}\,#3#1{).}\,#4}

% \fune is like \@fune but provides a standard interface.
%
% #1 : A formatting macro for symbols, such as \tfontsym.
% #2 : A formatting macro for text, such as \tfont.
% #3 : The pre-formatted variable the function binds.
% #4 : The pre-formatted type of the variable.
% #5 : The pre-formatted body of the function.
\newcommand{\fune}[2]{\@fune{#1}}

% \@polyfune formats a polymorphic function expression.
%
% #1 : A formatting macro for symbols, such as \tfontsym.
% #2 : The pre-formatted type variable the function binds.
% #3 : The pre-formatted variable the function binds.
% #4 : The pre-formatted type of the variable.
% #5 : The pre-formatted body of the function.
\newcommand{\@polyfune}[5]{
  #1{\uplambda}\,#1{[}{#2}#1{]\,(}{#3}\,#1{:}\,#4#1{).}\,#5
}

% \polyfune is like \@polyfune but provides a standard interface.
%
% #1 : A formatting macro for symbols, such as \tfontsym.
% #2 : A formatting macro for text, such as \tfont.
% #3 : The pre-formatted type variable the function binds.
% #4 : The pre-formatted variable the function binds.
% #5 : The pre-formatted type of the variable.
% #6 : The pre-formatted body of the function.
\newcommand{\polyfune}[2]{\@polyfune{#1}}

% \@abstre formats an polymorphic abstraction expression.
%
% #1 : A formatting macro for symbols, such as \tfontsym.
% #2 : The pre-formatted type variable the abstraction binds.
% #3 : The pre-formatted the body of the abstract.
\newcommand{\@abstre}[3]{#1{\Uplambda}#2#1{.}#3}

% \abstre is like \@abstre but provides a standard interface.
%
% #1 : A formatting macro for symbols, such as \tfontsym.
% #2 : A formatting macro for text, such as \tfont.
% #3 : The pre-formatted type variable the abstraction binds.
% #4 : The pre-formatted the body of the abstract.
\newcommand{\abstre}[2]{\@abstre{#1}}

% \@inste formats an instantiation expression.
%
% #1 : A formatting macro for symbols, such as \tfontsym.
% #2 : The pre-formatted polymorphic abstraction to instantiate.
% #3 : The pre-formatted type with which to instantiate the abstraction.
\newcommand{\@inste}[3]{#2\:#1{[}{#3}#1{]}}

% \inste is like \@inste but provides a standard interface.
%
% #1 : A formatting macro for symbols, such as \tfontsym.
% #2 : A formatting macro for text, such as \tfont.
% #3 : The pre-formatted polymorphic abstraction to instantiate.
% #4 : The pre-formatted type with which to instantiate the abstraction.
\newcommand{\inste}[2]{\@inste{#1}}

% \@app formats an application expression.
%
% #1 : The pre-formatted function.
% #2 : The pre-formatted argument.
\newcommand{\@appe}[2]{#1\ #2}

% \appe is like \@appe but provides a standard interface.
%
% #1 : A formatting macro for symbols, such as \tfontsym.
% #2 : A formatting macro for text, such as \tfont.
% #3 : The pre-formatted function.
% #4 : The pre-formatted argument.
\newcommand{\appe}[2]{\@appe}

% \@pappe formats a polymorphic function application expression.
%
% #1 : A formatting macro for symbols, such as \tfontsym.
% #2 : A pre-formatted polymorphic function expression.
% #3 : A pre-formatted type with which to instantiate.
% #4 : A pre-formatted argument to the function.
\newcommand{\@pappe}[4]{#2\:#1{[}{#3}#1{]}\:#4}

% \pappe is like \@pappe but provides a standard interface.
%
% #1 : A formatting macro for symbols, such as \tfontsym.
% #2 : A formatting macro for text, such as \tfont.
% #3 : A pre-formatted polymorphic function expression.
% #4 : A pre-formatted type with which to instantiate.
% #5 : A pre-formatted argument to the function.
\newcommand{\pappe}[2]{\@pappe{#1}}

% \@ife formats an if expression 
%
% #1 : A formatting macro for symbols, such as \tfontsym.
% #2 : A pre-formatted discriminant expression.
% #3 : A pre-formatted consequent expression.
% #4 : A pre-formatted alternate expression.
\newcommand{\@ife}[4]{#1{if}\ {#2}\ #1{then}\ {#3}\ #1{else}\ #4}

% \ife is like \@ife but provides a standard interface.
%
% #1 : A formatting macro for symbols, such as \tfontsym.
% #2 : A formatting macro for text, such as \tfont.
% #3 : A pre-formatted discriminant expression.
% #4 : A pre-formatted consequent expression.
% #5 : A pre-formatted alternate expression.
\newcommand{\ife}[2]{\@ife{#2}}

% \@packe formats a pack expression
%
% #1 : A formatting macro for text, such as \tfont.
% #2 : A formatting macro for symbols, such as \tfontsym.
% #3 : A pre-formatted type witness.
% #4 : A pre-formatted value witness.
% #5 : A pre-formatted existential type abstracting the witness type.
\newcommand{\@packe}[5]{
  #1{pack}#2{\langle} #3 #1{,} #4 #2{\rangle}\,#1{as}\,#5
}

% \packe is like \@packe but provides a standard interface.
%
% #1 : A formatting macro for symbols, such as \tfontsym.
% #2 : A formatting macro for text, such as \tfont.
% #3 : A pre-formatted type witness.
% #4 : A pre-formatted value witness.
% #5 : A pre-formatted existential type abstracting the witness type.
\newcommand{\packe}[2]{\@packe{#2}{#1}}

% \@unpacke formats an unpack expression
%
% #1 : A formatting macro for text, such as \tfont.
% #2 : A formatting macro for symbols, such as \tfontsym.
% #3 : A pre-formatted type variable.
% #4 : A pre-formatted value variable.
% #5 : A pre-formatted existential witness expression.
% #6 : A pre-formatted expression in which to bind the existential.
\newcommand{\@unpacke}[6]{
  #1{unpack}#2{\langle}#3 #1{,} #4 #2{\rangle\! = \,} #5 \,#1{in}\, #6
}

% \unpacke is like \@unpacke but provides a standard interface.
%
% #1 : A formatting macro for symbols, such as \tfontsym.
% #2 : A formatting macro for text, such as \tfont.
% #3 : A pre-formatted type variable.
% #4 : A pre-formatted expression variable.
% #5 : A pre-formatted existential witness expression.
% #6 : A pre-formatted expression in which to bind the existential.
\newcommand{\unpacke}[2]{\@unpacke{#2}{#1}}

% \@lete formats a let expression
%
% #1 : A formatting macro for text, such as \tfont.
% #2 : A formatting macro for symbols, such as \tfontsym.
% #3 : A pre-formatted expression variable.
% #4 : A pre-formatted expression to bind.
% #5 : A pre-formatted body expression for the let.
\newcommand{\@lete}[5]{#1{let}\ #3\:#2{=}\:{#4}\ #1{in}\ #5}

% \lete is like \@lete but provides a standard interface
%
% #1 : A formatting macro for symbols, such as \tfontsym.
% #2 : A formatting macro for text, such as \tfont.
% #3 : A pre-formatted expression variable.
% #4 : A pre-formatted expression to bind.
% #5 : A pre-formatted body expression for the let.
\newcommand{\lete}[2]{\@lete{#2}{#1}}

% \@folde formats a fold expression.
%
% #1 : A formatting macro for text, such as \tfont.
% #2 : A pre-formatted mu type.
% #3 : A pre-formatted expression of isorecursive type.
\newcommand{\@folde}[3]{#1{fold}_{#2} \, #3}

% \folde is like \@folde but provides a standard interface.
%
% #1 : A formatting macro for symbols, such as \tfontsym.
% #2 : A formatting macro for text, such as \tfont.
% #3 : A pre-formatted mu type.
% #4 : A pre-formatted expression of isorecursive type.
\newcommand{\folde}[2]{\@folde{#2}}

% \@unfolde formats an unfold expression
%
% #1 : A formatting macro for text, such as \tfont.
% #2 : A pre-formatted expression of isorecursive type.
\newcommand{\@unfolde}[2]{#1{unfold} \, #2}

% \unfolde is like \@unfolde but provides a standard interface.
%
% #1 : A formatting macro for symbols, such as \tfontsym.
% #2 : A formatting macro for text, such as \tfont.
% #3 : A pre-formatted expression of isorecursive type.
\newcommand{\unfolde}[2]{\@unfolde{#2}}

% \@unite formats a unit expression.
%
% #1 : A formatting macro for symbols, such as \tfontsym.
\newcommand{\@unite}[1]{#1{\langle}#1{\rangle}}

% \unite is like \@unite but provides a standard interface.
%
% #1 : A formatting macro for symbols, such as \tfontsym.
% #2 : A formatting macro for text, such as \tfont.
\newcommand{\unite}[2]{\@unite{#1}}

% \@truee formats a true expression
%
% #1 : A formatting macro for text, such as \tfont.
\newcommand{\@truee}[1]{#1{true}}

% \truee is like \@truee but provides a standard interface.
%
% #1 : A formatting macro for symbols, such as \tfontsym.
% #2 : A formatting macro for text, such as \tfont.
\newcommand{\truee}[2]{\@truee{#2}}

% \@falsee formats a false expression
%
% #1 : A formatting macro for text, such as \tfont.
\newcommand{\@falsee}[1]{#1{false}}

% \falsee is like \@falsee but provides a standard interface.
%
% #1 : A formatting macro for symbols, such as \tfontsym.
% #2 : A formatting macro for text, such as \tfont.
\newcommand{\falsee}[2]{\@falsee{#2}}

% \@paire formats a pair expression.
%
% #1 : A formatting macro for symbols, such as \tfontsym.
% #2 : A pre-formatted expression for the first component of the pair.
% #3 : A pre-formatted expression for the second component of the pair.
\newcommand{\@paire}[3]{#1{\langle}#2#1{,}#3{#1{\rangle}}}

% \paire is like \@paire but provides a standard interface.
%
% #1 : A formatting macro for symbols, such as \tfontsym.
% #2 : A formatting macro for text, such as \tfont.
% #3 : A pre-formatted expression for the first component of the pair.
% #4 : A pre-formatted expression for the second component of the pair.
\newcommand{\paire}[2]{\@paire{#1}}

% \@prje formats a projection expression.
%
% #1 : A formatting macro for symbols, such as \tfontsym.
% #2 : A formatting macro for text, such as \tfont.
% #3 : An unformatted natural number, either 1 or 2, indicating which
%   component of the pair to project.
% #4 : A pre-formatted pair expression to project.
\newcommand{\@prje}[4]{#1{\pi}_{#2{#3}} \, #4}

% \prje is like \@prje but provides a standard interface.
%
% #1 : A formatting macro for symbols, such as \tfontsym.
% #2 : A formatting macro for text, such as \tfont.
% #3 : An unformatted natural number, either 1 or 2, indicating which
%   component of the pair to project.
% #4 : A pre-formatted pair expression to project.
\newcommand{\prje}{\@prje}

% \@sume formats a sum expression.
%
% #1 : A formatting macro for text, such as \tfont.
% #2 : An unformatted natural number, either 1 or 2, indicating into which
%   side of the sum to inject the expression. 1 indicates left, 2
%   indicates right.
% #3 : A pre-formatted expression to inject into a sum.
\newcommand{\@sume}[3]{#1{inj}_{#1{#2}} \, #3}

% \sume is like \@sume but provides a standard interface.
%
% #1 : A formatting macro for symbols, such as \tfontsym.
% #2 : A formatting macro for text, such as \tfont.
% #3 : An unformatted natural number, either 1 or 2, indicating into which
%   side of the sum to inject the expression. 1 indicates left, 2
% #4 : A pre-formatted expression to inject into a sum.
\newcommand{\sume}[2]{\@sume{#2}}

% \@casee formats a case expression.
%
% #1 : A formatting macro for text, such as \tfont.
% #2 : A formatting macro for symbols, such as \tfontsym.
% #3 : A pre-formatted discriminant expression.
% #4 : A pre-formatted variable to bind in the left branch.
% #5 : A pre-formatted expression for the left branch.
% #6 : A pre-formatted variable to bind in the right branch.
% #7 : A pre-formatted expression for the right branch.
\newcommand{\@casee}[7]{
  #1{case} \, #3 \, #1{of} \, #4#2{.}\,#5\:#2{|}\:#6#2{.}\,#7
}

% \casee is like \@casee but provides a standard interface
%
% #1 : A formatting macro for symbols, such as \tfontsym.
% #2 : A formatting macro for text, such as \tfont.
% #3 : A pre-formatted discriminant expression.
% #4 : A pre-formatted variable to bind in the left branch.
% #5 : A pre-formatted expression for the left branch.
% #6 : A pre-formatted variable to bind in the right branch.
% #7 : A pre-formatted expression for the right branch.
\newcommand{\casee}[2]{\@casee{#2}{#1}}

% \newconstr generates a macro for formatting a language construct.
%
% #1 : A formatting macro for symbols, such as \tfontsym
% #2 : A formatting macro for text, such as \tfont
% #3 : A construct tag, such as fun
% #4 : A construct type, either ty or e
% #5 : The name of the macro to generate
\newcommand{\newconstr}[5]{
  \@globalshadowedcommand{#5}{0}{
    \csname#3#4\endcsname{#1}{#2}
  }
}

% \@newconstrs generates formatting macros for a list of language
% constructs
%
% #1 : A formatting macro for symbols, such as \tfontsym
% #2 : A formatting macro for text, such as \tfont
% #3 : A construct kind, either ty or e
% #4 : A prefix to use when generating the macro names
% #5 : A suffix to use when generating the macro names
% #6 : A list of construct tags, such as {fun,bool,unit,void}
\newcommand{\@newconstrs}[6]{
  {
    \foreach \@n in {#6}{
      \edef\@meow{\noexpand\newconstr{\noexpand#1}{\noexpand#2}{\@n}{#3}{#4\@n#5}}
      \@meow
    }
    \def\@meow{\undefined}
    \def\@n{\undefined}
  }
}

% \newtypes generates type formatting macros given a list of tags.
%
% #1 : A formatting macro for symbols, such as \tfontsym
% #2 : A formatting macro for text, such as \tfont
% #3 : A prefix for the name of each macro, such as t
% #4 : A suffix for the name of each macro, such as ty
% #5 : A list of type tags, such as {fun,bool,void,unit}
\newcommand{\newtypes}[2]{\@newconstrs{#1}{#2}{ty}}

% \newexprs generates type formatting macros given a list of tags.
%
% #1 : A formatting macro for symbols, such as \sfontsym
% #2 : A formatting macro for text, such as \sfont
% #3 : A prefix for the name of each macro, such as s
% #4 : A suffix for the name of each macro, such as e
% #5 : A list of type tags, such as {fun,bool,void,unit}
\newcommand{\newexprs}[2]{\@newconstrs{#1}{#2}{e}}
\makeatother

% \newlanguage generates macros for formatting meta-variables, types,
% and expressions of a language.
%
% #1 : A formatting macro for super-scripts, sub-scripts, and primes in
%   this language, such as \tcolor
% #2 : A formatting macro for math text in this language, such as \tfont
% #3 : A formatting macro for symbols in this language, such as
%   \tfontsym
% #4 : A language prefix for the macros, such as t
% #5 : A list of meta-variables for which to generate macros via
%   \newmetavars, such as {x, e, v}
% #6 : A list of meta-variables for which to generate macros via
%   \newmetavarsS, such as {ty/\tau,alpha\alpha}
% #7 : A list of types for which to generate macros via \newtypes, such
%   as {fun, forall, exist, pair, bool, unit}. The macros are generated
%   with the language prefix and the suffix ty.
% #8 : A list of expressions for which to generate macros via
%   \newexprs, such as {fun, app, if, true, false, unit}. The macros
%   are generated with the language prefix and the suffix e.
\renewcommand{\newlanguage}[8]{
  \newmetavars{#2}{#1}{#4}{#5}
  \newmetavarsS{#3}{#1}{#4}{#6}
  \newtypes{#3}{#2}{#4}{ty}{#7}
  \newexprs{#3}{#2}{#4}{e}{#8}
}

%%% Standard meta-theory

% \wf formats a well-formedness judgment.
%
% #1 : Pre-formatted assumptions, such as \tfont{\Delta}
% #2 : The pre-formatted proposition, such as \tfont{\alpha}
\newcommand{\wf}[2]{#1 \vdash #2}

% \judg formats a well-typedness judgment.
% It takes 3 parameters.
%
% #1 : The assumptions, such as \tfont{\Delta};\tfont{\Gamma}
% #2 : The term, such as \tfont{e}
% #3 : The type, such as \tfont{\tau}
\newcommand{\judg}[3]{\wf{#1}{#2 : #3}}

%% Context typing

% \ctxtarrowty formats a context typing arrow
%
% #1 : A formatting macro for symbols, such as \stfontsym
% #2 : The type of the hole
% #3 : The type of the result
\newcommand{\ctxtarrowty}[3]{#2 \, #1{\ctxarrow} \, #3}

% \ctxtty formats a context type, with different typing contexts for the
% hole and result
%
% #1 : A formatting macro for symbols, such as \stfontsym
% #2 : The typing contexts for the hole
% #3 : The type of the hole
% #4 : The typing contexts for the result
% #5 : The type of the result
\newcommand{\ctxtty}[5]{\ctxtarrowty{#1}{(\wf{#2}{#3})}{(\wf{#4}{#5})}}

% \ctxttyjudg formats a context typing judgment.
%
% #1 : A formatting macro for symbols, such as \stfontsym
% #2 : A pre-formatted context
% #3 : The typing contexts for the hole
% #4 : The type of the hole
% #5 : The typing contexts for the result
% #6 : The type of the result
\newcommand{\ctxttyjudg}[2]{\judg{}{#2}{\ctxtty{#1}}}


%% Contextual equivalence
\newcommand{\ctxeqvsym}{\approx^{\mit{ctx}}}
\newcommand{\ciueqvsym}{\approx^{\mit{ciu}}}

% \ciueqvjudg formats a c.i.u. equivalence judgment.
%
% #1 : Pre-formatted typing contexts, such as
%   \tfont{\Delta};\tfont{\Gamma}
% #2 : A pre-formatted expression
% #3 : A pre-formatted c.i.u. equivalent expression
% #4 : A pre-formatted type for the expressions
\newcommand{\ciueqvjudg}[4]{\judg{#1}{#2\,\ciueqvsym\,#3}{#4}}

% \ctxeqvjudg formats a contextual equivalence judgment.
%
% #1 : Pre-formatted typing contexts, such as
%   \tfont{\Delta};\tfont{\Gamma}
% #2 : A pre-formatted expression
% #3 : A pre-formatted contextually equivalent expression
% #4 : A pre-formatted type for the expressions
\newcommand{\ctxeqvjudg}[4]{\judg{#1}{#2\,\ctxeqvsym\,#3}{#4}}

%% Logical Relations
\newcommand{\lratomsym}{\mrm{Atom}}
\newcommand{\lratomvalsym}{\mrm{Atom}^{\mrm{val}}}
\newcommand{\lrrelsym}{\mrm{Rel}}

% \lr formats a logical relation set.
%
% #1 : A formatting macro, such as \tfont
% #2 : A letter for the relation, such a V or E
% #3 : A pre-formatted index for the relation
%
% Usage:
%  \newcommand{\srelV}{\lr{\sfont}{V}}
%  \newcommand{\trelV}{\lr{\tfont}{V}}
\newcommand{\lr}[3]{#1{\mathcal{#2}}\sembrace{#3}}

% \lrV,\lrE,\lrG,\lrD,\lrK,\lrO format logical relation sets.
%
% #1 : A formatting macro, such as \tfont
% #2 : A pre-formatted index for relation
%
% Usage:
%  \newcommand{\srelV}{\lrV{\sfont}}
%  \newcommand{\trelV}{\lrV{\tfont}}
\newcommand{\lrV}[2]{\lr{#1}{V}{#2}}
\newcommand{\lrE}[2]{\lr{#1}{E}{#2}}
\newcommand{\lrG}[2]{\lr{#1}{G}{#2}}
\newcommand{\lrD}[2]{\lr{#1}{D}{#2}}
\newcommand{\lrK}[2]{\lr{#1}{K}{#2}}
\newcommand{\lrO}[2]{\lr{#1}{O}{#2}}

%% The following help define formatting macros relational
%% interpretations and substitution, normally written as \rho, \delta,
%% and \gamma

%% TODO Maybe these should use \pair and \triple

% \binmapext extends a map whose value is a pair.
% It takes 4 parameters:
% #1 : A pre-formatted symbol for the map, such as \tfont{\rho}
% #2 : A pre-formatted key for the new mapping, such as \tfont{\alpha}
% #3 : A pre-formatted symbol for the first element of the value such as
%   \tfont{\tau}
% #4 : A pre-formatted symbol for the second element of the value
%
% Usage:
%  \newcommand{\slrgamma}{\sfont{\gamma}}
%  \newcommand{\slrgammaext}{\binmapext{\slrgamma}}
\newcommand{\binmapext}[4]{#1[#2 \mapsto (#3,#4)]}

% \trimapext is like \binmapext, but extends a map whose value is a triple.
% It takes 5 parameters:
%
% #1 : A pre-formatted symbol for the map, such as \tfont{\rho}
% #2 : A pre-formatted key for the new mapping, such as \tfont{\alpha}
% #3 : A pre-formatted symbol for the first element of the value such as
%   \tfont{\tau}
% #4 : A pre-formatted symbol for the second element of the value
% #5 : A pre-formatted symbol for the third element of the value
%
% Usage:
%  \newcommand{\tlrrho}{\tfont{\rho}}
%  \newcommand{\tlrrhoext}{\trimapext{\tlrrho}}
\newcommand{\trimapext}[5]{#1[#2 \mapsto (#3,#4,#5)]}

%% TODO These should probably accept formatting macros for the
%% subscript.

% \maponeat projects the first element of the value of a map at some
% key.
% It takes 2 arguments.
%
% #1 : A pre-formatted symbol for the map, as as \tfont{\rho}
% #2 : A pre-formatted symbol for the key
\newcommand{\maponeat}[2]{{#1}_1({#2})}

% \maptwoat is like \maponeat but projects the second element.
% The interface is the same
\newcommand{\maptwoat}[2]{{#1}_2({#2})}

% \maprelat is like \maponeat but projects the third element, which is
% assumed to be a relation.
\newcommand{\maprelat}[2]{{#1}_R({#2})}

%% Standard Source/Target formatting
\makeatletter
\newcommand*{\slang}{\@source-language-undefined}
\newcommand*{\tlang}{\@target-language-undefined}
\makeatother

\newcommand{\scolor}[1]{\textcolor{blue}{#1}}
\newcommand{\tcolor}[1]{\textcolor{red}{#1}}

\newcommand{\sfonttext}[1]{\textsf{\scolor{#1}}}
\newcommand{\tfonttext}[1]{\textsf{\tcolor{#1}}}

\newcommand{\sfont}[1]{\mathsf{\scolor{#1}}}
\newcommand{\tfont}[1]{\mathbf{\tcolor{#1}}}

\newcommand{\sfontsym}[1]{\sfont{#1}}
\newcommand{\tfontsym}[1]{\b{\tfont{#1}}}

\newcommand{\scal}[1]{\sfontsym{\mathcal{#1}}}
\newcommand{\tcal}[1]{\tfontsym{\mathcal{#1}}}

\newcommand{\sprime}{\scolor{\prime}}
\newcommand{\tprime}{\tcolor{\prime}}

