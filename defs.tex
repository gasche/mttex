\newcommand{\todoleft}[2][]{
  \reversemarginpar
  \todo[#1]{#2}
  \reversemarginpar
}

\newcommand{\margincomment}[2][color=green!40,nolist]{
  \todo[#1]{#2}
}

\newcommand{\inlinecomment}[2][inline,color=green!40,nolist]{
  \todo[#1]{#2}
}

\newcommand{\appendixcomment}[1]{#1}

%% Misc commands <<<

\newcommand{\secref}[1]{\S\ref{#1}}
\newcommand{\figref}[1]{Figure~\ref{#1}}
\newcommand{\lemref}[1]{Lemma~\ref{#1}}
\newcommand{\thmref}[1]{Theorem~\ref{#1}}

% Text fonts
\newcommand{\tbf}[1]{\textbf{#1}}
\newcommand{\trm}[1]{\textrm{#1}}
\newcommand{\tnl}[1]{\small{\trm{#1}}}

% Math fonts
\newcommand{\mbb}[1]{\mathbb{#1}}
\newcommand{\mbf}[1]{\mathbf{#1}}
\renewcommand{\mit}[1]{\mathit{#1}}
\newcommand{\mrm}[1]{\mathrm{#1}}
\newcommand{\mtt}[1]{\mathtt{#1}}
\newcommand{\mcal}[1]{\mathcal{#1}}
\newcommand{\mfrak}[1]{\mathfrak{#1}}
\newcommand{\msf}[1]{\mathsf{#1}}
\newcommand{\mscr}[1]{\mathscr{#1}}
\renewcommand{\b}{\boldsymbol}

% Text mode
\newenvironment{nop}{}{}

% Math mode
\newenvironment{sdisplaymath}{
\begin{nop}\small\begin{displaymath}}{
\end{displaymath}\end{nop}\ignorespacesafterend}
\newenvironment{fdisplaymath}{
\begin{nop}\footnotesize\begin{displaymath}}{
\end{displaymath}\end{nop}\ignorespacesafterend}
\newenvironment{smathpar}{
\begin{nop}\small\begin{mathpar}}{
\end{mathpar}\end{nop}\ignorespacesafterend}
\newenvironment{fmathpar}{
\begin{nop}\footnotesize\begin{mathpar}}{
\end{mathpar}\end{nop}\ignorespacesafterend}
\newenvironment{alignS}{
\begin{nop}\begin{align}}{
\end{align}\end{nop}\ignorespacesafterend}
\newenvironment{salignS}{
\begin{nop}\small\begin{align}}{
\end{align}\end{nop}\ignorespacesafterend}
\newenvironment{falignS}{
\begin{nop}\footnotesize\begin{align*}}{
\end{align}\end{nop}\ignorespacesafterend}

% Stack formatting
\newenvironment{stackAux}[2]{%
\setlength{\arraycolsep}{0pt}
\begin{array}[#1]{#2}}{
\end{array}}
\newenvironment{stackCC}{
\begin{stackAux}{c}{c}}{\end{stackAux}}
\newenvironment{stackCL}{
\begin{stackAux}{c}{l}}{\end{stackAux}}
\newenvironment{stackTL}{
\begin{stackAux}{t}{l}}{\end{stackAux}}
\newenvironment{stackTR}{
\begin{stackAux}{t}{r}}{\end{stackAux}}
\newenvironment{stackBC}{
\begin{stackAux}{b}{c}}{\end{stackAux}}
\newenvironment{stackBL}{
\begin{stackAux}{b}{l}}{\end{stackAux}}

\newtheoremstyle{athm}{\topsep}{\topsep}%
     {\itshape}%         Body font
     {}%         Indent amount (empty = no indent, \parindent = para indent)
     {\bfseries}% Thm head font
     {}%        Punctuation after thm head
%     {.8em}%     Space after thm head (\newline = linebreak)
     {\newline}%     Space after thm head (\newline = linebreak)
     {\thmname{#1}\thmnumber{ #2}\thmnote{~\,(#3)}}%         Thm head spec

\theoremstyle{athm}
\newtheorem{theorem}{Theorem}[section]
\newtheorem{lemma}[theorem]{Lemma}
\newtheorem{conjecture}[theorem]{Conjecture}
\newtheorem{corollary}[theorem]{Corollary}
\newtheorem{definition}[theorem]{Definition}

%% TODO I don't know if I want any of these.
%Theorems
% \newcounter{thrmi}
% \newcounter{lemi}
% \newcounter{defni}
% \newcounter{prfi}

% %\newcommand{\qed}{\hfill$\Box$}

% % \newtheorem{theoremX}{Theorem}
% \newtheorem{theoremX}{Theorem}[section]
% \newenvironment{theorem}[1][]{%
% \begin{theoremX}~%
% \ifthenelse{\equal{#1}{}}{}{\textup{\textbf{(#1)}}}~%
% \begin{list}{}{}\item[]}{%
% \end{list}\end{theoremX}}
% \newenvironment{theoremE}[1][]{%
% \begin{theoremX}~%
% \ifthenelse{\equal{#1}{}}{}{\textup{\textbf{(#1)}}}~%
% \begin{list}{\arabic{thrmi}.}{\usecounter{thrmi}}\item[]}{%
% \end{list}\end{theoremX}}

% \newtheorem{lemmaX}[theoremX]{Lemma}
% \newenvironment{lemma}[1][]{%
% \begin{lemmaX}~%
% \ifthenelse{\equal{#1}{}}{}{\textup{\textbf{(#1)}}}~%
% \begin{list}{}{}\item[]}{%
% \end{list}\end{lemmaX}}
% \newenvironment{lemmaE}[1][]{%
% \begin{lemmaX}~%
% \ifthenelse{\equal{#1}{}}{}{\textup{\textbf{(#1)}}}~%
% \begin{list}{\arabic{lemi}.}{\usecounter{lemi}}}{%
% \end{list}\end{lemmaX}}

% \newtheorem{conjectureX}[theoremX]{Conjecture}
% \newenvironment{conjecture}[1][]{%
% \begin{conjectureX}~%
% \ifthenelse{\equal{#1}{}}{}{\textup{\textbf{(#1)}}}~%
% \begin{list}{}{}\item[]}{%
% \end{list}\end{conjectureX}}
% \newenvironment{conjectureE}[1][]{%
% \begin{conjectureX}~%
% \ifthenelse{\equal{#1}{}}{}{\textup{\textbf{(#1)}}}~%
% \begin{list}{\arabic{lemi}.}{\usecounter{lemi}}}{%
% \end{list}\end{conjectureX}}

% \newtheorem{corollaryX}[theoremX]{Corollary}
% \newenvironment{corollary}[1][]{%
% \begin{corollaryX}~%
% \ifthenelse{\equal{#1}{}}{}{\textup{\textbf{(#1)}}}~%
% \begin{list}{}{}\item[]}{%
% \end{list}\end{corollaryX}}
% \newenvironment{corollaryE}[1][]{%
% \begin{corollaryX}~%
% \ifthenelse{\equal{#1}{}}{}{\textup{\textbf{(#1)}}}~%
% \begin{list}{\arabic{thrmi}.}{\usecounter{thrmi}}}{%
% \end{list}\end{corollaryX}}

% \newtheorem{definitionX}[theoremX]{Definition}
% \newenvironment{definition}[1][]{%
% \begin{definitionX}~%
% \ifthenelse{\equal{#1}{}}{}{\textup{\textbf{(#1)}}}~%
% \begin{list}{}{}\item[]}{%
% \end{list}\end{definitionX}}
% \newenvironment{definitionE}[1][]{%
% \begin{definitionX}~%
% \ifthenelse{\equal{#1}{}}{}{\textup{\textbf{(#1)}}}~%
% \begin{list}{\arabic{defni}.}{\usecounter{defni}}}{%
% \end{list}\end{definitionX}}
%
% \newenvironment{proof}[1][]{%
% \noindent%
% \textbf{Proof\ifthenelse{\equal{#1}{}}{}{ (#1)}}%
% \begin{list}{}{}\item[]}{%
% \qed
% \end{list}}
% \newenvironment{proofE}[1][]{%
% \noindent%
% \textbf{Proof\ifthenelse{\equal{#1}{}}{}{ (#1)}}%
% \begin{list}{\arabic{prfi}.}{\usecounter{prfi}}}{%
% \qed
% \end{list}}

\renewenvironment{proof}[1][]{%
\noindent%
\textbf{Proof\ifthenelse{\equal{#1}{}}{}{ (#1)}}%
\begin{list}{}{}\item[]}{%
\qed
\end{list}}

\renewenvironment{cases}{\begin{description}}{\end{description}}
\newcommand{\case}[1][]{\item[Case] #1}
\newcommand{\scase}[1][]{\item[Sub-case] #1}
\newcommand{\sscase}[1][]{\item[SubSub-case] #1}

\newsavebox{\DetailsBox}%
\newenvironment{details}[1][]{%
\ifthenelse{\boolean{ShowDetailsFlag}}{}{#1 \begin{lrbox}{\DetailsBox}\begin{minipage}{\textwidth}}%
}{%
\ifthenelse{\boolean{ShowDetailsFlag}}{}{\end{minipage}\end{lrbox}}%
}%

%% TODO Or these
%% \renewcommand{\thefigure}{\thesection.\arabic{figure}}

% % Shading
% \newlength{\ShadeLen}
% \newcommand{\shade}[2]{%
% \settoheight{\ShadeLen}{$\bnfalt$}
% \colorbox[gray]{#1}{\mbox{\rule{0pt}{\ShadeLen}#2}}}

%% Meta Language
\newcommand{\metafont}[1]{\mathrm{#1}}
\newcommand{\dom}[1]{\metafont{dom}(#1)}
\newcommand{\cod}[1]{\metafont{cod}(#1)}
\newcommand{\rng}[1]{\metafont{rng}(#1)}
\newcommand{\FV}{\metafont{fv}}
\newcommand{\FTV}{\metafont{ftv}}
\newcommand{\subst}[3]{{#1}[{#2}/{#3}]}
\newcommand{\defeq}{\stackrel{\metafont{def}}{=}}
\newcommand{\finmap}{\stackrel{\metafont{fin}}{\rightarrow}}
\renewcommand{\iff}{\metafont{iff}}
\newcommand{\lsem}{\left\llbracket}
\newcommand{\rsem}{\right\rrbracket}
\newcommand{\sembrace}[1]{\lsem{#1}\rsem}
\newcommand{\powset}[1]{\mathscr{P}(#1)}
\newcommand{\irred}[1]{\metafont{irred}(#1)}
\renewcommand{\max}[2]{\metafont{max}(#1,#2)}
\newcommand{\free}[2]{\metafont{free}(#1,#2)}
\newcommand{\running}[2]{\metafont{running}({#1},{#2})}
\newcommand{\pair}[2]{( #1, #2 )}
\newcommand{\triple}[3]{( #1, #2, #3 )}

\newcommand{\satisfy}{\vDash}

\newcommand{\plus}{+}
\newcommand{\etal}{\textit{et al.}}
\newcommand{\bump}{\hspace{3.5pt}}
\newcommand{\fresh}[1]{(\mit{fresh}\:#1)}
\newcommand{\where}[1]{\mrm{where}\:#1}

\newcommand{\lang}[1]{\mrm{\trm{#1}}}

%% Standard symbols

\newcommand{\empctx}{\cdot}
\newcommand{\hole}{[\cdot]}
\newcommand{\hw}[1]{\lbrack{#1}\rbrack}

\newcommand{\hooklongrightarrow}{\lhook\joinrel\longrightarrow}
\newcommand{\redexstep}{\hookrightarrow}

\newcommand{\step}{\longmapsto}
\newcommand{\stepin}[1]{\step^{#1}}
\newcommand{\stepstar}{\stepin{*}}

\newcommand{\red}{\Downarrow}
\newcommand{\diverg}[1]{#1 \Uparrow}

\newcommand{\termin}[1]{#1\red}
\newcommand{\terminw}[2]{\termin{#1} #2}

\newcommand{\transarrow}{\leadsto}
\newcommand{\backtransarrow}{\twoheadrightarrow}

\newcommand{\funarrow}{\rightarrow}
\newcommand{\ctxarrow}{\Rightarrow}

% BNF symbols
\newcommand{\bnfalt}{{\bf \,\,\mid\,\,}}
\newcommand{\bnfdef}{{\bf ::=}}

%% Language Formatting combinators

% Copy-pasta: http://tex.stackexchange.com/questions/16189/repeat-command-n-times
\makeatletter
\newcommand{\Repeat}[1]{%
  \expandafter\@Repeat\expandafter{\the\numexpr #1\relax}%
}

\def\@Repeat#1{%
  \ifnum#1>0
    \expandafter\@@Repeat\expandafter{\the\numexpr #1-1\expandafter\relax\expandafter}%
  \else
    \expandafter\@gobble
  \fi
}
\def\@@Repeat#1#2{%
  \@Repeat{#1}{#2}{#2}
}
\makeatother

% \lmetavar formats a language meta-var. It takes 4 parameters:
% #1 : a pre-formatted symbol representing the meta-var
% #2 : a pre-formatted subscript
% #3 : a pre-formatted superscript
% #4 : a pre-formatted prime symbol
% #5 : a natural number, representing the number of primes
%
% Usage:
%   \newcommand{\txmetavar}[3]{
%     \lmetavar{\tfont{x}}{\tcolor{#1}}{\tcolor{#2}}{\tprime}{#3}
%   }
%   \newcommand{\tx}{\txmetavar{}{}{}}
%   \newcommand{\txone}{\txmetavar{1}{}{}}
%   \newcommand{\txonepr}{\txmetavar{1}{}{1}}

\newcommand{\lmetavar}[5]{
  {#1}_{#2}^{{#3}\Repeat{#5}{#4}}
}

% \lmetavarto formats a language meta-var with only a
% superscript, and takes 2 parameters:
%
% #1 : a pre-formatted symbol representing the meta-var
% #2 : a pre-formatted superscript
%
% Usage:
%   \newcommand{\txF}{\lmetavarto{\tx}{f}}
\newcommand{\lmetavarto}[2]{\lmetavar{#1}{}{#2}{}{0}}

% \lmetavarin formats a language meta-var with only a
% subscript, takes 2 parameters:
%
% #1 : a pre-formatted symbol representing the meta-var
% #2 : a pre-formatted subscript
%
% Usage:
%   \newcommand{\txone}{\lmetavarto{\tx}{\tcolor{1}}}
\newcommand{\lmetavarin}[2]{\lmetavar{#1}{#2}{}{}{0}}

% \lmetavarpr formats a language meta-var with only primes, takes 3
% parameters:
%
% #1 : a pre-formatted symbol representing the meta-var
% #2 : a pre-formatted prime symbol
% #3 : a natural number representing the number of primes
%
% Usage:
%   \newcommand{\txpr}{\lmetavarto{\tx}{\prime}{1}}
%   \newcommand{\txdubpr}{\lmetavarto{\tx}{\prime}{2}}
\newcommand{\lmetavarpr}[3]{\lmetavarpr{#1}{}{}{#2}{#3}}

% \newlmetavar* generates some standard macros for formatting a meta-var,
% and takes 5 paraemters:
%
% #1 : A formatting macro for the meta-var, such a \tfont
% #2 : A formatting macro for the subscripts, superscripts, and primes,
%   such as \tcolor
% #3 : A string prefix for the name of the macro, such as t
% #4 : A string name of the macro, such as ty
% #5 : A string to format and display this meta-variable as, such as
%   \tau
%
% Usage:
%   \newlmetavar*{\tfont}{\tcolor}{t}{x}{x}
% This usage would generates definitions like:
%  \newcommand{\tx}{\tfont{x}}
%  \newcommand{\txmetavar}[3]{
%    \lmetvar{\tx}{\tcolor{#1}}{\tcolor{#2}}{\tcolor{\prime}}{#3}
%  }
%  \newcommand{\txin}[1]{ \txmetavar{#1}{}{0} }
%  \newcommand{\txto}[1]{ \txmetavar{}{#1}{0} }
%  \newcommand{\txpr}{\tx^{\tcolor{\prime}}}
%  \newcommand{\txdubpr}{\tx^{\tcolor{\prime\prime}}}
%  \newcommand{\txone}{\tx_\tcolor{1}}
%  \newcommand{\txonepr}{\txone^{\tcolor{\prime}}}
%  \newcommand{\txtwo}{\tx_\tcolor{2}}
%  \newcommand{\txtwopr}{\txtwo^{\tcolor{\prime}}}

\makeatletter

\newcommand{\@globalshadowedcommand}[4][]{
  {
    \let\def\gdef
    \ifcsname#2\endcsname
    \message{
      ^^J^^J
      !!!!!!!!!!!!!!!!!!!!!!!!!!!!!!!!!!!!!!!!!!!!!!!!!!!!!!!!!!!!!!^^J
      !! secctex Warning: shadowed \@backslashchar#2 when defining metavar macros.^^J
      !! If you need this macro, use \@backslashchar LetLtxMacro to save it.^^J
      !!!!!!!!!!!!!!!!!!!!!!!!!!!!!!!!!!!!!!!!!!!!!!!!!!!!!!!!!!!!!!^^J
      ^^J
    }
    \let\newcommand\renewcommand
    \fi
    \if\relax\detokenize{#1}\relax
    \expandafter\newcommand\csname #2\endcsname[#3]{#4}
    \else
    \expandafter\newcommand\csname #2\endcsname[#3][#1]{#4}
    \fi
  }
}

\newcommand{\newlmetavarStar}[5]{
  \@globalshadowedcommand{#3#4metavar}{3}{
    \lmetavar{#1{#5}}{#2{##1}}{#2{##2}}{#2{\prime}}{##3}
  }
  \@globalshadowedcommand{#3#4}{0}{
    \csname #3#4metavar\endcsname{}{}{0}
  }
  \@globalshadowedcommand{#3#4in}{1}{
    \csname #3#4metavar\endcsname{##1}{}{0}
  }
  \@globalshadowedcommand[0]{#3#4to}{2}{
    \csname #3#4metavar\endcsname{}{##2}{##1}
  }
  \@globalshadowedcommand[1]{#3#4pr}{1}{
    \csname #3#4metavar\endcsname{}{}{##1}
  }
  \@globalshadowedcommand{#3#4one}{0}{
    \csname #3#4metavar\endcsname{1}{}{0}
  }
  \@globalshadowedcommand[1]{#3#4onepr}{1}{
    \csname#3#4metavar\endcsname{1}{}{##1}
  }
  \@globalshadowedcommand{#3#4two}{0}{
    \csname #3#4metavar\endcsname{2}{}{0}
  }
  \@globalshadowedcommand[1]{#3#4twopr}{1}{
    \csname #3#4metavar\endcsname{2}{}{##1}
  }
  \@globalshadowedcommand{#3#4n}{0}{
    \csname #3#4metavar\endcsname{n}{}{0}
  }
  \@globalshadowedcommand[1]{#3#4npr}{1}{
    \csname #3#4metavar\endcsname{n}{}{##1}
  }
  \@globalshadowedcommand{#3#4i}{0}{
    \csname #3#4metavar\endcsname{i}{}{0}
  }
  \@globalshadowedcommand[1]{#3#4ipr}{1}{
    \csname #3#4metavar\endcsname{i}{}{##1}
  }
}

\newcommand{\newlmetavarNoStar}[4]{
  \newlmetavarStar{#2}{#3}{#4}{#1}{#1}
}

\newcommand{\newlmetavarmaybestar}{
  \@ifstar
  \newlmetavarStar
  \newlmetavarNoStar
}

\newcommand{\newlmetavar}[4]{
  \newlmetavarmaybestar#4{#1}{#2}{#3}
}

% TODO Use \@ for more internal vars and such
% TODO DOCUMENT
% Usage:
%   \newlmetavars{\tfont}{\tcolor}{t}{x/x,e/e,v/v,alpha/\tfontsym{\alpha}}
\newcommand{\newlmetavarsS}[4]{
  \foreach \@n/\@d in {#4} {
    \edef \@meow{\noexpand\newlmetavarStar{\noexpand#1}{\noexpand#2}{#3}{\@n}{\@d}}
    \@meow
  }
}

\newcommand{\newlmetavars}[4]{
  \foreach \@n in {#4} {
    \edef \@meow{\noexpand\newlmetavarStar{\noexpand#1}{\noexpand#2}{#3}{\@n}{\@n}}
    \@meow
  }
}

\makeatother

%% Standard meta-vars

\newcommand{\ectxt}{E}
\newcommand{\ctxt}{C}

%%% Standard language features
%%% TODO Document. Tests and examples exist in test.tex

%% Standard types
\newcommand{\funty}[3]{#2 #1{\funarrow} \, #3}
\newcommand{\funtyS}[4]{\funty{#1}{#3}{#4}}

\newcommand{\polyfunty}[4]{#1{\forall}\,#1{[}{#2}#1{].}{#3}#1{\funarrow}\,{#4}}
\newcommand{\polyfuntyS}[5]{\polyfunty{#1}{#3}{#4}{#5}}

\newcommand{\forallty}[3]{#1{\forall}#2#1{.}\,#3}
\newcommand{\foralltyS}[4]{\forallty{#1}{#3}{#4}}

\newcommand{\existty}[3]{#1{\exists}#2#1{.}\,#3}
\newcommand{\existtyS}[4]{\existty{#1}{#3}{#4}}

\newcommand{\muty}[3]{#1{\upmu} #2 #1{.} #3 }
\newcommand{\mutyS}[4]{\muty{#1}{#3}{#4}}

\newcommand{\unitty}[1]{#1{1}}
\newcommand{\unittyS}[2]{\unitty{#2}}

\newcommand{\voidty}[1]{#1{0}}
\newcommand{\voidtyS}[2]{\voidty{#2}}

\newcommand{\boolty}[1]{#1{bool}}
\newcommand{\booltyS}[2]{\boolty{#2}}

\newcommand{\pairty}[3]{#2 \, #1{\times} \, #3}
\newcommand{\pairtyS}[4]{\pairty{#1}{#3}{#4}}

\newcommand{\sumty}[3]{#2 \, #1{\plus} \, #3}
\newcommand{\sumtyS}[4]{\sumty{#1}{#3}{#4}}

%% Standard expression
\newcommand{\fune}[4]{#1{\uplambda (}{#2}\,#1{:}\,#3#1{).}\,#4}
\newcommand{\funeS}[5]{\fune{#1}{#3}{#4}{#5}}

\newcommand{\polyfune}[5]{
  #1{\uplambda}\,#1{[}{#2}#1{]\,(}{#3}\,#1{:}\,#4#1{).}\,#5
}
\newcommand{\polyfuneS}[6]{\polyfune{#1}{#3}{#4}{#5}{#6}}

\newcommand{\abstre}[3]{#1{\Uplambda}#2#1{.}#3}
\newcommand{\abstreS}[4]{\abstre{#1}{#3}{#4}}

\newcommand{\inste}[3]{#2\:#1{[}{#3}#1{]}}
\newcommand{\insteS}[4]{\inste{#1}{#3}{#4}}

\newcommand{\appe}[2]{#1\ #2}
\newcommand{\appeS}[4]{\appe{#3}{#4}}

\newcommand{\pappe}[4]{#2\:#1{[}{#3}#1{]}\:#4}
\newcommand{\pappeS}[5]{\pappe{#1}{#3}{#4}{#5}}

\newcommand{\ife}[4]{#1{if}\ {#2}\ #1{then}\ {#3}\ #1{else}\ #4}
\newcommand{\ifeS}[5]{\ife{#2}{#3}{#4}{#5}}

\newcommand{\packe}[5]{
  #1{pack}#2{\langle} #3 #1{,} #4 #2{\rangle}\,#1{as}\,#5
}
\newcommand{\packeS}[5]{\packe{#2}{#1}{#3}{#4}{#5}}

\newcommand{\unpacke}[6]{
  #1{unpack}#2{\langle}#3 #1{,} #4 #2{\rangle\! = \,} #5 \,#1{in}\, #6
}
\newcommand{\unpackeS}[6]{\unpacke{#2}{#1}{#3}{#4}{#5}}

\newcommand{\lete}[5]{#1{let}\ #3\:#2{=}\:{#4}\ #1{in}\ #5}
\newcommand{\leteS}[5]{\lete{#2}{#1}{#3}{#4}{#5}}

\newcommand{\folde}[3]{#1{fold}_{#2} \, #3}
\newcommand{\foldeS}[4]{\folde{#2}{#1}{#3}{#4}}

\newcommand{\unfolde}[2]{#1{unfold} \, #2}
\newcommand{\unfoldeS}[3]{\unfolde{#2}{#3}}

\newcommand{\unite}[1]{#1{\langle}#1{\rangle}}
\newcommand{\uniteS}[2]{\unite{#1}}

\newcommand{\truee}[1]{#1{true}}
\newcommand{\trueeS}[2]{\truee{#2}}

\newcommand{\falsee}[1]{#1{false}}
\newcommand{\falseeS}[2]{\falsee{#2}}

\newcommand{\paire}[3]{#1{\langle}#2#1{,}#3{#1{\rangle}}}
\newcommand{\paireS}[4]{\paire{#1}{#3}{#4}}

\newcommand{\prje}[3]{#1{\pi}_{#2{i}} \, #3}
\newcommand{\prjeS}{\prje}

\newcommand{\sume}[3]{#1{inj}_{#1{#2}} \, #3}
\newcommand{\sumeS}[4]{\sume{#2}{#3}{#4}}

\newcommand{\casee}[7]{
  #1{case} \, #3 \, #1{of} \, #4#2{.}\,#5\:#2{|}\:#6#2{.}\,#7
}
\newcommand{\caseeS}[7]{\casee{#2}{#1}{#3}{#4}{#5}{#6}{#7}}

\makeatletter
\newcommand{\newlconstr}[4]{
  \@globalshadowedcommand{#3#4}{0}{
    \csname#4S\endcsname{#1}{#2}
  }
}

\newcommand{\newlconstrs}[5]{
  {
    \foreach \@n in {#5}{
      \edef\@meow{\noexpand\newlconstr{\noexpand#1}{\noexpand#2}{#3}{\@n#4}}
      \@meow
    }
  }
}
\makeatother

\newcommand{\newltypes}[3]{\newlconstrs{#1}{#2}{#3}{ty}}
\newcommand{\newlexprs}[3]{\newlconstrs{#1}{#2}{#3}{e}}

\renewcommand{\newlanguage}[8]{
  \newlmetavars{#2}{#1}{#4}{#5}
  \newlmetavarsS{#3}{#1}{#4}{#6}
  \newltypes{#3}{#2}{#4}{#7}
  \newlexprs{#3}{#2}{#4}{#8}
}

%% Standard meta-theory

% \wf formats a well-formedness judgment.
% It takes 2 parameters.
%
% #1 : The assumptions, such as \tfont{\Delta}
% #2 : The proposition, such as \tfont{\alpha}
\newcommand{\wf}[2]{#1 \vdash #2}

% \judg formats a well-typedness judgment.
% It takes 3 parameters.
%
% #1 : The assumptions, such as \tfont{\Delta};\tfont{\Gamma}
% #2 : The term, such as \tfont{e}
% #3 : The type, such as \tfont{\tau}
\newcommand{\judg}[3]{#1 \vdash #2 : #3}

\newcommand{\ctxttysh}[3]{#2 #1{\ctxarrow} \, #3}
\newcommand{\ctxty}[5]{\ctxttysh{#1}{(\wf{#2}{#3})}{(\wf{#4}{#5})}}

\newcommand{\ctxeqv}{\approx^{\mit{ctx}}}
\newcommand{\ciueqv}{\approx^{\mit{ciu}}}

\newcommand{\lratomsym}{\mrm{Atom}}
\newcommand{\lratomvalsym}{\mrm{Atom}^{\mrm{val}}}
\newcommand{\lrrelsym}{\mrm{Rel}}

% \lr formats a logical relation set. It takes 3 parameters:
%
% #1 : A formatting macro, such as \tfont
% #2 : A letter for the relation, such a V or E
% #3 : A pre-formatted index for the relation
%
% Usage:
%  \newcommand{\srelV}[1]{\lr{\sfont}{V}{#1}}
%  \newcommand{\trelV}[2]{\lr{\tfont}{V}{#1}{#2}}
\newcommand{\lr}[3]{#1{\mathcal{#2}}\sembrace{#3}}

% \lrV,\lrE,\lrG,\lrD,\lrK,\lrO format logical relation set. They takes
% 2 parameters:
%
% #1 : A formatting macro, such as \tfont
% #2 : A pre-formatted index for relation
%
% Usage:
%  \newcommand{\srelV}[1]{\lrV{\sfont}{#1}}
%  \newcommand{\trelV}[2]{\lrV{\tfont}{#1}{#2}}
\newcommand{\lrV}[2]{\lr{#1}{V}{#2}}
\newcommand{\lrE}[2]{\lr{#1}{E}{#2}}
\newcommand{\lrG}[2]{\lr{#1}{G}{#2}}
\newcommand{\lrD}[2]{\lr{#1}{D}{#2}}
\newcommand{\lrK}[2]{\lr{#1}{K}{#2}}
\newcommand{\lrO}[2]{\lr{#1}{O}{#2}}

%% The following help define formatting macros relational
%% interpretations and substitution, normally written as \rho, \delta,
%% and \gamma

%% TODO Maybe these should use \pair and \triple

% \binmapext extends a map whose value is a pair.
% It takes 4 parameters:
% #1 : A pre-formatted symbol for the map, such as \tfont{\rho}
% #2 : A pre-formatted key for the new mapping, such as \tfont{\alpha}
% #3 : A pre-formatted symbol for the first element of the value such as
%   \tfont{\tau}
% #4 : A pre-formatted symbol for the second element of the value
%
% Usage:
%  \newcommand{\slrgamma}{\sfont{\gamma}}
%  \newcommand{\slrgammaext}[3]{\binmapext{\slrgamma}{#1}{#2}{#3}}
\newcommand{\binmapext}[4]{#1[#2 \mapsto (#3,#4)]}

% \trimapext is like \binmapext, but extends a map whose value is a triple.
% It takes 5 parameters:
%
% #1 : A pre-formatted symbol for the map, such as \tfont{\rho}
% #2 : A pre-formatted key for the new mapping, such as \tfont{\alpha}
% #3 : A pre-formatted symbol for the first element of the value such as
%   \tfont{\tau}
% #4 : A pre-formatted symbol for the second element of the value
% #5 : A pre-formatted symbol for the third element of the value
%
% Usage:
%  \newcommand{\tlrrho}{\tfont{\rho}}
%  \newcommand{\tlrrhoext}[4]{\trimapext{\tlrrho}{#1}{#2}{#3}{#4}}
\newcommand{\trimapext}[5]{#1[#2 \mapsto (#3,#4,#5)]}

%% TODO These should probably accept formatting macros for the
%% subscript.

% \maponeat projects the first element of the value of a map at some
% key.
% It takes 2 arguments.
%
% #1 : A pre-formatted symbol for the map, as as \tfont{\rho}
% #2 : A pre-formatted symbol for the key
\newcommand{\maponeat}[2]{{#1}_1({#2})}

% \maptwoat is like \maponeat but projects the second element.
% The interface is the same
\newcommand{\maptwoat}[2]{{#1}_2({#2})}

% \maprelat is like \maponeat but projects the third element, which is
% assumed to be a relation.
\newcommand{\maprelat}[2]{{#1}_R({#2})}

%% Standard Source/Target formatting
\makeatletter
\newcommand*{\slang}{\@source-language-undefined}
\newcommand*{\tlang}{\@target-language-undefined}
\makeatother

\newcommand{\scolor}[1]{\textcolor{blue}{#1}}
\newcommand{\tcolor}[1]{\textcolor{red}{#1}}

\newcommand{\sfonttext}[1]{\textsf{\scolor{#1}}}
\newcommand{\tfonttext}[1]{\textsf{\tcolor{#1}}}

\newcommand{\sfont}[1]{\mathsf{\scolor{#1}}}
\newcommand{\tfont}[1]{\mathbf{\tcolor{#1}}}

\newcommand{\sfontsym}[1]{\sfont{#1}}
\newcommand{\tfontsym}[1]{\b{\tfont{#1}}}

\newcommand{\scal}[1]{\sfontsym{\mathcal{#1}}}
\newcommand{\tcal}[1]{\tfontsym{\mathcal{#1}}}

\newcommand{\sprime}{\scolor{\prime}}
\newcommand{\tprime}{\tcolor{\prime}}

